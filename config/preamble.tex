\usepackage{setspace, fancyhdr, lscape, floatrow, caption, inputenc, graphicx, enumitem, tabularx, colorprofiles, xstring, hyphenat, chngcntr, xspace}

\usepackage[left=3cm,right=2.5cm,top=2.5cm,bottom=2cm]{geometry}
\usepackage[backend=biber,style=apa,citestyle=apa]{biblatex}
\usepackage[autostyle]{csquotes}
\usepackage[english, nswissgerman]{babel}
\usepackage[titles]{tocloft}

% Settings for glossary
\usepackage[acronym,toc=true,xindy,nopostdot=true,nonumberlist,nogroupskip=true]{glossaries}
\loadglsentries{content/01_vorspann/01.4_abkuerzungsverzeichnis_glossar}
\setglossarystyle{alttree}
\setglossarypreamble[acronym]{\vspace*{-8pt}}
\makeglossaries
\glsaddall
\glsfindwidesttoplevelname

\graphicspath{{content/00_assets}} % Set path as default for including graphics
\addbibresource{content/00_assets/quellen.bib} % Add bibliography entries

% cite with acronym. 
% Example input: \citeAbbr{schweizerische_archivdirektorinnen-_und_archivdirektorenkonferenz_informations-_2018}{adk}
\newcommand{\parenciteabbr}[2]{(\acrlong{#2} [\acrshort{#2}], \citeyear{#1})}
\newcommand{\textciteabbr}[2]{\acrlong{#2} (\acrshort{#2}, \citeyear{#1})}
\newcommand{\acrfullSqBr}[1]{\acrlong{#1} [\acrshort{#1}]}

% Settings for blockquote
\SetBlockEnvironment{quotation}
\patchcmd{\quotation}{\rightmargin}{\leftmargin 1.3cm \rightmargin 0}{}{}
\NewCommandCopy{\oldblockquote}{\blockquote}
\renewcommand{\blockquote}[1]{\oldblockquote{\noindent #1}}

% Settings for figure and table
\counterwithout{figure}{chapter}
\counterwithout{table}{chapter}

\floatsetup[figure]{
    capposition=above,
    captionskip=2pt
}

\DeclareCaptionLabelFormat{figTitle}{%
    Abbildung \the\numexpr\value{figure}+1\relax\\
}

\captionsetup[figure]{
    font={normalsize,stretch=1.0},
    labelfont=bf,
    labelsep=none,
    labelformat=figTitle,
    textfont=it,
    justification=raggedright,
    singlelinecheck=false,
    position=above,
}

\floatsetup[table]{
    capposition=above,
    captionskip=2pt
}

\DeclareCaptionLabelFormat{tabTitle}{%
    Tabelle \the\numexpr\value{table}+1\relax \\
} 

\captionsetup[table]{
    font={normalsize,stretch=1.0},
    labelfont={bf},
    labelsep=none,
    labelformat=tabTitle,
    textfont=it,
    justification=raggedright,
    singlelinecheck=false,
    position=above
}

\renewcommand{\thetable}{\arabic{table}}
\renewcommand{\thefigure}{\arabic{figure}}

% Einstellungen für tocloft / Settings for tocloft
\renewcommand{\cftfigpresnum}{Abb.~}
\renewcommand{\cfttabpresnum}{Tab.~}
\renewcommand{\cftfigaftersnum}{:}
\renewcommand{\cfttabaftersnum}{:}
\setlength{\cftfignumwidth}{2cm}
\setlength{\cfttabnumwidth}{2cm}
\setlength{\cftfigindent}{0cm}
\setlength{\cfttabindent}{0cm}

\setstretch{1.3} % Zeilenabstand / line spacing
\renewcommand{\arraystretch}{1.3} % Zeilenabstand innerhalb von Tabellen / line spacing in tables
\setlength{\parindent}{1.3cm} % Einzug neuer Absatz / Indentation new paragraph
\fancyhf{}
\renewcommand{\headrulewidth}{0pt}
\pagestyle{fancyplain}
\rfoot{\footnotesize\thepage}

\usepackage[pdfa]{hyperref}
\usepackage{hyperxmp}
\usepackage{embedfile}
\pdfvariable omitcidset=1

% 
\newcommand{\chapterNoNr}[1]{%
    \chapter*{#1}
    \addcontentsline{toc}{chapter}{#1} %
}%

\newcommand{\note}[1]{%
    \raggedright\footnotesize{\textit{Anmerkung:} #1}
}%

\newcommand{\sic}{%
    [\textit{sic}]\xspace
}%

\hypersetup{%
    pdflang=\sprache,
    pdftitle={\haupttitel},
    pdfsubtitle={\untertitel},
    pdfauthor={\autorenschaft},
    pdfdate={\abgabedatumRFC},
    pdfsubject={\zusammenfassung},
    pdfkeywords={\modul, \schlagworte},
    pdfcontactaddress={\adresse},
    pdfcontactcity={\ort},
    pdfcontactpostcode={\plz},
    pdfcontactemail={\email},
    colorlinks,
    unicode,
    allcolors=black,
    pdfapart=2,
    pdfaconformance=U
}

% 
\immediate\pdfobj stream attr{/N 3} file{sRGB.icc}
\pdfcatalog{%
  /OutputIntents [
    <<
      /Type /OutputIntent
      /S /GTS_PDFA1
      /DestOutputProfile \the\pdflastobj\space 0 R
      /OutputConditionIdentifier (sRGB)
      /Info (sRGB)
    >>
  ]
}
