\chapter{Erstes Kapitel}
\label{kap:eins}
Dies ist eine Vorlage für wissenschaftliche Arbeiten an der FHGR nach \textcite{stepanenko_leitfaden_2023}\footnote{Quelle aus dem Intranet (nicht öffentlich zugänglich) von FHGR}. Nachfolgend einige Tipps und Funktionen, welche in der Vorlage enthalten sind.

\section{Glossar und Abkürzungen}
Mit den Befehlen \setminus acrlong\{zb\}, \setminus acrshort\{zb\} oder \setminus acrfull\{zb\} ist das Ausgeben der im Dokument conten/00\_assets/01\_Vorspann/01.4\_abkuerzungsverzeichnis\_glossar.tex Einträgen möglich: \acrshort{zb}, \acrshort{zb}, \acrfull{zb}.

\section{Zitieren}
Für das Zitieren wird die Erweiterung BibLaTeX verwendet. Sämtliche Befehle und Funktionen des Pakets können im Handbuch nachgeschlagen werden\footnote{https://mirror.metanet.ch/tex-archive/info/translations/biblatex/de/biblatex-de-Benutzerhandbuch.pdf}. Voraussetzung ist, dass die Quellen erst im BibLaTeX Format unter content/00\_assets/quellen.bib abgelegt sind (z. B. via Zotero, Mendely oder als Datei). Nachfolgend Beispiele mit den üblichsten Befehlen:\\
\\
\setminus textcite\{diani\_black\_2016\} gibt als Ausgabe:\\ 
\textcite{diani_black_2016}\\
\\
\setminus parencite\{diani\_black\_2016\} gibt als Ausgabe:\\ 
\parencite{diani_black_2016}\\
\\
\setminus parencites[S. 10]\{diani\_black\_2016\}[S. 20]\{stepanenko\_leitfaden\_2023\} gibt als Ausgabe:\\ 
\parencites[S. 10]{diani_black_2016}[S. 20]{stepanenko_leitfaden_2023}\\
\\
Zusätzlich stehen die folgenden eigens im Template definierten Befehle zur Verfügung:\\
\\
\setminus parenciteabbr\{org\_2022\}\{orln\} gibt als Ausgabe:\\ 
\parenciteabbr{org_2022}{orln}\\
\\
\setminus textciteabbr\{org\_2022\}\{orln\} gibt als Ausgabe:\\ 
\textciteabbr{org_2022}{orln}\\
\\
Zur Formatierung eines Blockzitats kann der Befehl \setminus begin\{quote\} ... \setminus end\{quote\} verwendet werden:
\blockquote{Blockzitate sind wörtliche Zitate von 40 Wörtern oder mehr; sie werden als eigener Absatz ohne Anführungszeichen angeführt. Ein Blockzitat beginnt stets in einer neuen Zeile, wird zur Gänze (also jede Zeile) 1,3 cm oder fünf Leerschritte eingerückt und mit zweizeiligem Abstand geschrieben. Absätze innerhalb eines Blockzitates werden vom neuen Rand des Blockzitates eingerückt. \par Blockzitate werden nicht in Anführungszeichen gesetzt, darin aufscheinende Zitate werden in doppelten Anführungszeichen wiedergegeben. \par Die Quellenangabe am Ende eines Blockzitates steht nach dem letzten schließenden Punkt des Zitates in Klammern gesetzt, danach folgt kein weiterer Punkt.\\
\parencite[S. 111]{psychologie_richtlinien_2016}}
Um in einem wörtlichen Zitat auf Fehler hinzuweisen kann der Befehl \setminus sic verwendet werden, welcher ein \sic einfügt. Beispiel "<Darüberhinaus \sic handelt es sich ...">


\chapter{Ein Bild}
\begin{figure}[H]
    \caption{Überschrift Abbildung 1}
    \includegraphics[width=\textwidth]{alpendohle.jpg}
    \note{\textcite[Abs. 1]{diani_black_2016}}
    \label{fig:alpendohle}
\end{figure}

\chapter{Tabelle}
Im folgenden (Tabelle \ref{tab:tabelle}) ist ein Beispiel einer Tabelle:
\begin{table}[ht]
    \caption{Überschrift Tabelle 1}
    \begin{tabularx}{\textwidth} {
        >{\raggedright\arraybackslash}X 
        >{\raggedleft\arraybackslash}X 
        >{\raggedleft\arraybackslash}X}
            \hline
            \multicolumn{3}{c}{\textbf{Beispieltabelle}}\\
            \hline
            \textbf{Linksbündig} & \textbf{Rechtsbündig} & \textbf{Rechtsbündig}\\
            \hline
            Lorem & N/A & N/A\\
            Ipsum & 1 499 & 8 512\\
            Dolor & 297 & N/A\\
            Sit & 1 053 & N/A\\
            \hline
            \textbf{Total} & 2 849 & 8 512\\
            \hline
    \end{tabularx}
    \medbreak
    \note{\textcite[Abs. 1]{diani_black_2016}}
    \label{tab:tabelle}
\end{table}

\chapter{Programmcode}
Programmcode ist immer öfter Bestandteil von Studienarbeiten. Die \textcite{American_Psychological_Association_2022} schreibt zur Formatierung von Passagen mit Porgrammcode: "<To present computer code, use a monospace font such as 10-point Lucida Console or 10-point Courier New">. Im Fliesstext könnte dies folgendermassen aussehen: Die Funktion \lstinline{print()} in der Programmiersprache Python wird meist dazu benutzt, Text auf dem Bildschirm auszugeben. Das folgende Beispiel \ref{code:hello} gibt die Zeichenkette "<Hello World"> aus.
\begin{lstlisting}[language=Python,caption=Überschrift Programmcode,label=code:hello] 
# hello_world.py
print("Hello World!")
\end{lstlisting}
\medbreak
\noindent
Weitere Konfigurationsmöglichkeiten für die Darstellung von Programmcode können der Dokumentation\footnote{http://texdoc.net/texmf-dist/doc/latex/listings/listings.pdf} des \texttt{lstlisting} Pakets entnommen werden. 